\documentclass{article}
\usepackage[brazilian]{babel}

% TODO: Make this different for two-sided printing than for two-up printing
% (two-up should have equal L/R margins)
\usepackage[paperwidth=5.83in, paperheight=8.27in, twoside, top=0.7in, bottom=0.7in, inner=0.5in, outer=0.9in]{geometry}

\usepackage{caption}
\usepackage{subcaption}
\usepackage{graphicx}
% The following packages are included for proper support for diacritics.
% \usepackage[utf8]{inputenc}
% \usepackage[T1]{fontenc}

\begin{document}
\author{James Babcock (jimrandomh@gmail.com)}
\author{Tradu\c{c}\~{a}o: Bruno Parga (brunoparga@gmail.com}
\title{Dia de Petrov}

% Ideas to incorporate {{{
%  X Move the Malthus bit to after three candles are lit (establishing the pattern)
%  X Incorporate a tree of extinct human relatives, in the section about evolution
%    Mention collapsed civilizations: Easter Island, that island hypothesized as being Atlantis, Mayans, etc
%    Work in a mention of Moloch
%    Discuss the post-WW2 measures as a way of ending on a positive note: formation of the UN, twinning, reaties
%  - Mention abandonment of nuclear powered rockets
%    Mention environmental successes
% }}}

% Quotes to maybe use {{{
%
% Through the years her work continued to yield surprising insights, such as
% the unsettling discovery that chimpanzees engage in primitive and brutal
% warfare. In early 1974, a “four-year war” began at Gombe, the first record
% of long-term “warfare” in nonhuman primates. Members of the Kasakela group
% systematically annihilated members of the “Kahama” splinter group. --Jane Goodall

% Most species do their own evolving, making it up as they go along, which is
% the way Nature intended. -–Terry Pratchett

% Most gods throw dice, but Fate plays chess, and you don’t find out til too
% late that he’s been playing with two queens all along. -–Terry Pratchett

% Man still bears in his bodily frame the indelible stamp of his lowly origin. –Charles Darwin
% }}}


\newcommand{\divider}{ %{{{
	% From http://tex.stackexchange.com/questions/32711/totally-sweet-horizontal-rules-in-latex
	\nointerlineskip \vspace{\baselineskip}
	\hspace{\fill}\rule{0.5\linewidth}{.7pt}\hspace{\fill}
	\par\nointerlineskip \vspace{\baselineskip}
} %}}}
\newcommand{\stagedir} [1] { %{{{
	\begin{itshape}
	#1
	\end{itshape}
} %}}}
\newcommand{\blockquote} [2] { %{{{
	\begin{center}
		\parbox{3.5in}{
			``#1''
			\begin{flushright}
				--- #2
			\end{flushright}
		}
	\end{center}
} %}}}
\newcommand{\blockquoteUnattributed} [1] { %{{{
	\begin{center}
		\parbox{3.5in}{
			``#1''
		}
	\end{center}
} %}}}
\newcommand{\blockspacing} [1] { %{{{
	\begin{center}
		\parbox{3.5in}{
			#1
		}
	\end{center}
} %}}}
\newcommand{\blockquoteUnmarked} [2] { %{{{
	\begin{center}
		\parbox{3.5in}{
			#1
			\begin{flushright}
				--- #2
			\end{flushright}
		}
	\end{center}
} %}}}
\newcommand{\poem} [2] { %{{{
	\begin{center}
		\parbox{3.5in}{
			#1
			\begin{flushright}
				--- #2
			\end{flushright}
		}
	\end{center}
} %}}}
\newcommand{\page} [1] { %{{{
	\divider
	#1
	\divider
	\newpage
} %}}}
\newcommand{\pageNoBottomDiv} [1] { %{{{
	\divider
	#1
	\newpage
} %}}}
\newcommand{\sidePage} [1] { %{{{
	\vspace*{0.2in}
	#1
	\newpage
} %}}}
\newcommand{\candelabrum} [1] { %{{{
	\begin{center}
		\includegraphics[width=3in]{#1}
	\end{center}
} %}}}
\newcommand{\candlePassing} { %{{{
	\begin{center}
		\includegraphics[width=2.5in]{images/candlepassing.png}
	\end{center}
} %}}}

% Don't indent paragraphs
\setlength{\parindent}{0cm}

\setlength{\parskip}{\baselineskip}

%%%%%%%%%%%%%%%%%%%%%%%%%%%%%%%%%%%%%%%%%%%%%%%%%%%%%%%%%%%%%%%%%%%%%%

% Front Matter
% Title page {{{
\pagenumbering{gobble}
\sidePage{

\begin{flushright}
\parbox{3in}{
	\begin{center}
		\Huge{Dia de Petrov}\newline
		\large{26 de setembro}\newline
		\large{Por James Babcock}\newline
		\small{Tradu\c{c}\~{a}o por Bruno Parga}\newline
	\end{center}
}
\end{flushright}

} % }}}
% What-is-this page 1 {{{
\pagenumbering{arabic}
\page{

    O Dia de Petrov \'{e} um evento anual no dia 26 de setembro, rememorando o 
incidente Petrov: um alarme falso no sistema de alerta r\'{a}pido soviético
que praticamente desencadeou uma guerra nuclear. O prop\'{o}sito do ritual \'{e}
tornar o risco catastr\'{o}fico e existencial emocionalmente saliente, colocando-o
num contexto hist\'{o}rico e oferecendo exemplos positivos e negativos de
como a humanidade lidou com ele. A atividade \'{e} desaconselh\'{a}vel para pessoas
fr\'{a}geis e n\~{a}o \'{e} destinada a principiantes; \'{e} direcionada \`{a}queles que j\'{a} sabem
o que \'{e} o risco existencial e catastr\'{o}fico, t\^{e}m algum conhecimento pr\'{e}vio
sobre quais s\~{a}o esses riscos, e acreditam - ao menos num n\'{\i}vel abstrato - 
que \'{e} importante impedir que esses riscos se concretizem.

\divider

Voc\^{e} vai precisar de:

\begin{itemize} \itemsep0pt \parskip0pt \parsep0pt
	\item Uma c\'{o}pia deste programa para cada participante
	\item Uma mesa com cadeiras para todos
	\item Um candelabro para 9 velas
	\item 8 velas e algo para acend\^{e}-las (isqueiro, f\'{o}sforos)
	\item Um extintor de inc\^{e}ndio de f\'{a}cil acesso, caso necess\'{a}rio
	\item Um bloco de fichas de arquivo ou notas autoadesivas, mais canetas
\end{itemize}

} % }}}
% What-is-this page 2 {{{
\page{

Por James Babcock, com contribui\c{c}\~{o}es de conte\'{u}do por Ben Landau-Taylor, Adia
Porter e Daniel Speyer e cita\c{c}\~{o}es de diversas fontes. Agradecimentos a Eliezer
Yudkowsky pela ideia de comemorar o Dia de Petrov, e a todos os participantes
de testes, organizadores de eventos e outros que possibilitaram este ritual.

Tradu\c{c}\~{a}o em portugu\^{e}s do Brasil por Bruno Parga, setembro de 2017.

} % }}}

% Introduction Section
% Intro page {{{
\page{

%\stagedir{Content warning: Engineered to evoke strong feelings of existential terror.}

\stagedir{Instru\c{c}\~{o}es como estas v\^{e}m em it\'{a}lico. Todo o restante do texto deve
ser lido em voz alta. Sempre que houver uma linha horizontal, \'{e} a vez da pr\'{o}xima
pessoa ler, seguindo em sentido hor\'{a}rio. Ao ler cita\c{c}\~{o}es, n\~{a}o \'{e} necess\'{a}rio
ler o nome e data no final.}

Hoje, 26 de setembro, \'{e} o Dia de Petrov. Em 1983, a est\'{o}ria da humanidade quase
terminou. Estamos reunidos aqui para recordar esse momento, e outros como ele.
Mas, para realmente sentir a imensid\~{a}o destes eventos, precisamos visit\'{a}-los em
seu contexto apropriado. Vamos contar a est\'{o}ria da hist\'{o}ria humana, come\c{c}ando
do come\c{c}o.

\divider

\blockquote{No in\'{\i}cio, o universo foi criado. Isso irritou profundamente muitas
pessoas e, no geral, foi encarado como uma p\'{e}ssima ideia.}{Douglas Adams}

\divider

Vamos pular o prel\'{u}dio de treze bilh\~{o}es de anos. Nossa est\'{o}ria come\c{c}a na idade
dos mitos, dos f\'{o}sseis e das lendas. Come\c{c}a com a inven\c{c}\~{a}o do fogo.

} %}}}
% Prometheus quote: Light candle 1 {{{
\page{
\blockquote{Sou quem roubou, ca\c{c}ada no oco duma cana, a fonte do fogo, que se
revelou, para a humanidade, mestre de todas as artes e tesouro inestim\'{a}vel.
Esse o pecado que resgato, pregado nestas cadeias ao relento.}{\'{E}squilo, "Prometeu Acorrentado"}

\candelabrum{images/candelabrum1.png}

\stagedir{Acenda a primeira vela, para representar a inven\c{c}\~{a}o do fogo. Aponte
a localiza\c{c}\~{a}o do extintor de inc\^{e}ndio mais pr\'{o}ximo, e ent\~{a}o reduza ou apague
todas as outras luzes do recinto.}

} %}}}

% Prehistory

% About Fire {{{
\page {

Os arque\'{o}logos se dividem em dizer que o fogo foi utilizado pela primeira vez
pelo "Homo erectus" ou pelo "Homo ergaster", em algum ponto entre 400 mil e
1.7 milh\~{o}es de anos atr\'{a}s. O cozimento dos alimentos deve ter possibilitado 
c\'{e}rebros maiores, que consumiam mais energia, dando caminho \`{a} evolu\c{c}\~{a}o de uma
intelig\^{e}ncia mais avan\c{c}ada e da linguagem.

\divider

\blockquote{A maioria das esp\'{e}cies cuidam da pr\'{o}pria evolu\c{c}\~{a}o, improvisando
conforme o caso, bem do jeito que a Natureza quis. E tudo isso \'{e} muito natural
e org\^{a}nico e sintonizado com os misteriosos ciclos do cosmos, que acredita que
n\~{a}o h\'{a} nada como milh\~{o}es de anos realmente frustrantes de tentativa e erro para
uma esp\'{e}cie ganhar fibra moral e, em alguns casos, melhorar a postura.}{Terry Pratchett}

} %}}}
% Family tree {{{
\sidePage {

\newcommand{\skull} [2] { %{{{
	\parbox{1.3in}{
		\begin{center}
			\includegraphics[width=1.2in]{#1}
			#2
		\end{center}
	}
} %}}}

\skull{images/H_ergaster.png}{"Homo ergaster"}
\skull{images/H_erectus.png}{"Homo erectus"}
\skull{images/H_florensiensis.png}{"Homo florensiensis"} \\
\skull{images/H_habilis.png}{"Homo habilis"}
\skull{images/H_heidelbergensis.png}{"Homo heidelbergensis"}
\skull{images/H_neanderthalis.png}{"Homo neanderthalis"} \\
\skull{images/H_rudolfensis.png}{"Homo rudolfensis"}


} %}}}
% CUT: Fire and speech {{{
%\page {
%
%Estimates place the invention of fire somewhere from 400,000 to 1.7 million
%years ago, while anatomically modern humans did not appear until 200,000 years
%ago.
%
%Fire is not, by itself, of particularly great importance. It grants protection
%from the cold of winter, and access to foods that would otherwise be inedible;
%but if it had stopped there, then the story of genus homo would have been
%unremarkable. But fire enabled, and favored, creatures with larger brains, and
%some time after that - we don't know when - our ancestors began to speak.
%
%} % }}}


% The invention of language: Candle 2 {{{
\pageNoBottomDiv{
%\blockquote{It's perfectly obvious that there is some genetic factor that
%distinguishes humans from other animals and that it is language-specific. The
%theory of that genetic component, whatever it turns out to be, is what is
%called universal grammar.}{Noam Chomsky}
% Source: http://www.slate.com/articles/health_and_science/new_scientist/2012/03/noam_chomsky_on_linguistics_and_climate_change_.html

\blockquote{Certamente n\~{a}o se trata de um verdadeiro instinto, pois cada idioma
deve ser aprendido. \'{E} nitidamente diferente, por\'{e}m, de todas as artes comuns,
pois as pessoas t\^{e}m uma tend\^{e}ncia instintiva a falar, como vemos no balbucio de
nossas crian\c{c}as pequenas; por contraste, nenhuma crian\c{c}a tem uma tend\^{e}ncia
instintiva a cozinhar, a fermentar bebidas ou a escrever.}{Charles Darwin,
A Descend\^{e}ncia do Homem (1871)}

} %}}}
% Language: Candle 2 {{{
\sidePage {

\stagedir{Pegue a primeira vela, que representa a inven\c{c}\~{a}o do fogo e use-a para
acender a segunda vela, que representa a evolu\c{c}\~{a}o da linguagem.}

\candlePassing

\stagedir{Passe a vela de m\~{a}o em m\~{a}o ao redor do c\'{\i}rculo. Quando voc\^{e}
estiver com a vela na m\~{a}o, \'{e} sua vez de falar. Qual \'{e} o seu nome, e de quando
(que ano) \'{e} a sua mem\'{o}ria mais antiga?

\candelabrum{images/candelabrum2.png}

Quando todos tiverem falado, ponha a vela de volta no candelabro.}

\divider

}
% }}}


% Agriculture: Candle 3 {{{
\page{

A linguagem \'{e} a primeira chave da tecnologia; com ela, os primeiros humanos
podiam acumular conhecimento n\~{a}o apenas em genes mas tamb\'{e}m em ditos e tradi\c{c}\~{o}es.

Eles deram nomes \`{a}s pessoas ao seu redor. Deram nomes a esp\'{e}cies de animais e
plantas. Deram nomes a a\c{c}\~{o}es, lugares e estrat\'{e}gias. Chamaram algumas dessas
coisas de 'boas', e chamaram outras de 'm\'{a}s'. Aprenderam a compartilhar seu
conhecimento, e aprenderam a enganar um ao outro. Constru\'{\i}ram fam\'{\i}lias e
comunidades.

Ent\~{a}o come\c{c}aram o longo e demorado processo de domar a natureza selvagem. Suas
tribos cresceram e se tornaram cidades. O que aconteceu depois?

\stagedir{Pegue a segunda vela, que representa a linguagem, e use-a para
acender a terceira vela, que representa a agricultura.}

\candelabrum{images/candelabrum3.png}

} %}}}
% Uplift {{{
\sidePage {

\stagedir{Se voc\^{e} ou outra pessoa presente conhecem a letra desta m\'{u}sica,
cantem; se n\~{a}o, leia normalmente.}

\begin{center}
	\parbox{2in}{
		\begin{large}Uplift\end{large}\newline
		Por Andrew Eigel
	}
\end{center}

\blockspacing{
	Hands chip the flint, light the fire, skin the kill\newline
	Feet move the tribe track the herd with a will\newline
	Mankind struggles in the cellar of history\newline
	Time to settle down, time to grow, time to breed\newline
	\newline
	Plow tills the soil, plants the seed, pray for rain\newline
	Scythe reaps the wheat, to the mill, to grind the grain\newline
	Towns and cities spread to empire overnight\newline
	Hands keep building as we chant the ancient rite}

\stagedir{Pare aqui. Siga para a pr\'{o}xima p\'{a}gina sem ler ou cantar o restante.}

\blockspacing{
	Coal heats the steam, push the piston, turns the wheel\newline
	Cogs spin the wool, drives the horses made of steel\newline
	Lightning harnessed does our will and lights the dark\newline
	Keep rising higher, set our goal, hit the mark.\newline
	\newline
	\-\hspace{0.5in}Crawl out of the mud,\newline
	\-\hspace{0.5in}Ongoing but slow,\newline
	\-\hspace{0.5in}For the path that is easy\newline
	\-\hspace{0.5in}Ain't the one that lets us grow!\newline
	\newline
	Light to push the sails, read the data, cities glow\newline
	Hands type the keys, click the mouse, out we go!\newline
	Our voices carry round the world and into space\newline
	Send us out to colonize another place¸\newline
	\newline
	Hands make the tools, build the fire, plant the grain.\newline
	Feet track the herd, build a world, begin again.}


}
% }}}


% The Malthusian Trap {{{
\pageNoBottomDiv {
\blockquote{O poder do povoamento \'{e} t\~{a}o superior ao poder da terra de produzir
o sustento do homem que a morte prematura deve, de uma maneira ou de outra,
visitar a ra\c{c}a humana. Os v\'{\i}cios da humanidade s\~{a}o diligentes e capazes
ministros do despovoamento. Eles s\~{a}o a vanguarda do grande ex\'{e}rcito da
destrui\c{c}\~{a}o, e muitas vezes terminam eles mesmos a macabra tarefa. Mas caso eles
falhem nesta guerra de exterm\'{\i}nio, esta\c{c}\~{o}es m\'{o}rbidas, epidemias, pestes e
pragas avan\c{c}am num terr\'{\i}vel pelot\~{a}o, varrendo milhares e dezenas de milhares.
Se o sucesso ainda for incompleto, a gigantesca e inevit\'{a}vel inani\c{c}\~{a}o se
esgueira na retaguarda, e com um forte golpe nivela a popula\c{c}\~{a}o com a comida
dispon\'{\i}vel no mundo.}{Thomas Malthus (1798)}
} %}}}
% The Malthusian Trap {{{
\sidePage {

\stagedir{Pegue a terceira vela, que representa a sociedade agr\'{\i}cola. Passe-a
ao redor do c\'{\i}rculo.

\candlePassing

Assopre-a e retorne-a, apagada, para o seu lugar no candelabro.}

\candelabrum{images/candelabrum3b.png}

\divider


}
% }}}


% Agriculture {{{
\page{

A humanidade vivia em equil\'{\i}brio entre o crescimento e a ru\'{\i}na, o conhecimento
ganho e o conhecimento esquecido. Nesse mundo, as est\'{o}rias tinham a dura\c{c}\~{a}o da
mem\'{o}ria, e monumentos tinham a dura\c{c}\~{a}o da madeira. Por duzentos mil anos, nada
sobreviveu exceto genes.

\divider

Mas isso foi o bastante. Embora n\~{a}o pudessem preservar o conhecimento de gera\c{c}\~{a}o
para gera\c{c}\~{a}o, podiam preservar plantas e animais domesticados. Guardavam os
melhores e, de pouco em pouco, o mundo ficou mais f\'{a}cil. Ent\~{a}o alguns distintos
humanos come\c{c}aram a escrever, e o equil\'{\i}brio entre aprendizado e esquecimento
foi finalmente rompido.

Dessa era, que mem\'{o}rias permanecem?

%This raised the population density, but the Malthusian limit soon caught up.
%The equilibrium between growth and starvation was not broken. Worse, farming
%brought bad diets, poor health, and numerous social ills.
%
%
%\blockquote{Archaeologists studying the rise of farming have reconstructed a crucial stage
%at which we made the worst mistake in human history. Forced to choose between
%limiting population or trying to increase food production, we chose the latter
%and ended up with starvation, warfare, and tyranny.}{Jared Diamond (1987)}
%
%Agriculture did not lift humanity out of the Malthusian trap; it raised the
%population ceiling up, at the expense of peoples' health and well being.
%
%Despite its problems, farming enabled the growth of civilizations, and the
%rulers of some of these civilizations came to realize that they did not want to
%be forgotten. Some built monuments. About 5,000 years ago, some began to write.

\vspace{0.4in}

\stagedir{Usando a segunda vela, que representa a linguagem, acenda novamente a
terceira vela para representar a inven\c{c}\~{a}o da escrita.}

\candelabrum{images/candelabrum3.png}

}
%}}}
% Invention of writing: Candle 3 again; write names of family members {{{
\page{

\poem{
	Conheci um viajante de uma terra ancestral\newline
	Contou-me: Sem tronco, duas pernas enormes\newline
	Erguem-se no deserto... Perto delas no areal,\newline
	Semienterrada, a cabe\c{c}a em partes disformes,\newline
	Franze o cenho, e o esc\'{a}rnio de um comando glacial,\newline
	Mostra-nos que o escultor captou bem o seu estado\newline
	Que ainda sobrevive estampado nessas pedras est\'{e}reis,\newline
	A m\~{a}o que dele tro\c{c}ou e o cora\c{c}\~{a}o que foi alimentado;\newline
	E no pedestal est\~{a}o grafadas as seguintes palavras:\newline
	“Meu nome \'{e} Ozim\^{a}ndias, rei dos reis:\newline
	\'{o} Poderosos, rendei-vos ao olhar minhas obras!”\newline
	Nada al\'{e}m permanece. Ao redor do desolamento\newline
	Da ru\'{\i}na colossal, infinitas e desertas\newline
	As areias planas e solit\'{a}rias se estendem ao vento.}{Percy Bysshe Shelley (1818), trad. Alberto Marsicano}

\stagedir{Ao terminar de ler, pegue um peda\c{c}o de papel e escreva o nome do
familiar mais antigo - vivo ou morto - que voc\^{e} consiga identificar.

Quando todos tiverem escrito algo, siga para a pr\'{o}xima p\'{a}gina.}

}
% }}}

% History

% Writing and The Rosetta Stone {{{
\page{

Sabe-se mais sobre como era o mundo depois que as pessoas come\c{c}aram a escrever,
mas n\~{a}o \'{e} muito o que restou. Uma das inscri\c{c}\~{o}es mais importantes foi descoberta
por soldados franceses na estrutura do Forte Julien: a Pedra de Roseta, cuja
import\^{a}ncia se deve a ser escrita em tr\^{e}s l\'{\i}nguas, duas delas at\'{e} ent\~{a}o
intraduz\'{\i}veis. Ap\'{o}s uma longa s\'{e}rie de honor\'{\i}ficos e decretos sobre impostos e
sucess\~{a}o, ela declara: fa\c{c}a-se um novo feriado!

\divider

\blockquote{Nestes dias de cada m\^{e}s, nos quais haver\'{a} sacrif\'{\i}cios e liba\c{c}\~{o}es e
todas as cerim\^{o}nias costumeiras de outros festivais, e as oferendas dever\~{a}o ser
dadas aos sacerdotes que vivem nos templos. E um festival deve ser institu\'{\i}do
para o rei Ptolomeu, O de vida eterna, o Amado de Ptah, o Deus Epifanes
Eucaristos, anualmente nos templos por toda a terra, por cinco dias a partir do
dia primeiro de Thoth ... Este decreto ser\'{a} inscrito numa estela de pedra dura
em hier\'{o}glifos, dem\'{o}tico e letras gregas, e promulgado em cada um dos tr\^{e}s
primeiros templos junto \`{a} imagem do rei de vida eterna.}{A pedra de Roseta (ca. 196 a.e.c.)}

} %}}}
% The Rosetta Stone image {{{
\sidePage{

\includegraphics[width=4in]{images/RosettaStone.png}

}
% }}}


% Scientific Method: Candle 4; write something surprising you learned {{{
\pageNoBottomDiv{

A maioria das obras escritas consistia de genealogias, c\'{o}digos legais, e
est\'{o}rias fant\'{a}sticas. Mas alguns dos textos representavam progresso na filosofia
e na matem\'{a}tica, culminando na inven\c{c}\~{a}o do m\'{e}todo cient\'{\i}fico.

\divider

\blockquote{A matem\'{a}tica \'{e} o port\~{a}o e a chave das ci\^{e}ncias... O desprezo \`{a}
matem\'{a}tica causa danos a todo o conhecimento, pois aquele que a ignora n\~{a}o tem
como conhecer as outras ci\^{e}ncias ou as coisas deste mundo. E o que \'{e} pior,
homens t\~{a}o ignorantes s\~{a}o incapazes de perceber sua pr\'{o}pria ignor\^{a}ncia e, assim,
n\~{a}o buscam remedi\'{a}-la.}{Roger Bacon, Opus Majus (1266)}

} %}}}
% Scientific method {{{
\sidePage{

\stagedir{
Usando a terceira vela, que representa a escrita, acenda a quarta vela para
representar o m\'{e}todo cient\'{\i}fico.

\candelabrum{images/candelabrum4.png}

% TODO: Replace this with a better written thing
Ent\~{a}o, todos escrevem algo surpreendente que tenham aprendido na \'{u}ltima semana,
e p\~{o}em o papel no centro. Quando todos tiverem escrito algo, siga para a
pr\'{o}xima p\'{a}gina.}

\divider

}
% }}}


% Black Death: Blow out candle 4 {{{
\page{

O m\'{e}todo cient\'{\i}fico, combinado com a escrita e um sistema de universidades,
marcou o princ\'{\i}pio de uma acumula\c{c}\~{a}o de conhecimento. Isto poderia ter dado
in\'{\i}cio \`{a} lenta transi\c{c}\~{a}o rumo \`{a} modernidade. Por\'{e}m, 81 anos ap\'{o}s Roger Bacon,
o trem da hist\'{o}ria descarrilou devido a uma grande peste.

\divider

\stagedir{Pegue a quarta vela, que representa o progresso da ci\^{e}ncia. Segure-a,
enquanto l\^{e} a cita\c{c}\~{a}o.}

\blockquote{No ano da era de mil trezentos e quarenta e oito anos veio a
pestil\^{e}ncia e a mortandade de dor de levadigas por todo o mundo, t\~{a}o grande que
l\'{a} n\~{a}o ficaram vivos um d\'{e}cimo dos homens e mulheres que l\'{a} ent\~{a}o havia. E no
dito ano morreram o prior, o chantre e todos os ra\c{c}oeiros da igreja de S\~{a}o Pedro
de Almedina de Coimbra, uns depois dos outros, em um m\^{e}s.}{Joaquim de Santa Rosa de Viterbo, Elucid\'{a}rio}

\stagedir{Assopre a vela, apagando-a. Ent\~{a}o retorne-a ao seu lugar no candelabro.}

\candelabrum{images/candelabrum4b.png}

} %}}}
% Black Death cont'd {{{
\page{

A praga matou cerca de metade da popula\c{c}\~{a}o da Europa num per\'{\i}odo de quatro anos,
e ela recorreu repetidamente nos tr\^{e}s s\'{e}culos seguintes, a cada vez matando dez
por cento da popula\c{c}\~{a}o ou mais. Entre pragas, guerras e fomes, havia pouco
tempo para preservar ou avan\c{c}ar o conhecimento.

} %}}}


% Printing press: Re-light candle 4 {{{
\page{

Preservar o conhecimento demandava redund\^{a}ncia. Em 1439, durante a Renascen\c{c}a
europeia, Gutenberg realizou uma inven\c{c}\~{a}o que tornava isso poss\'{\i}vel.

\divider

\blockquoteUnmarked{``Diga-me, amigo Martinho, quantas impress\~{o}es podem ser
feitas por esta prensa em um dia?'' ``Cerca de trezentas, se a operarmos
continuamente.'' ``Ser\'{a} poss\'{\i}vel!'' exclamou Pedro. ``Com efeito, agora os
livros h\~{a}o de se multiplicar. O que ser\'{a} que os laboriosos copistas dir\~{a}o a
respeito?''}{Emily Clemens Pearson, Gutenberg e a Arte da Impress\~{a}o (1870)}

\stagedir{Pegue a quarta vela, que representa o progresso da ci\^{e}ncia.

Toque com ela cada uma das outras tr\^{e}s velas, at\'{e} que ela se acenda. Ent\~{a}o
retorne-a a seu lugar no candelabro.}

\candelabrum{images/candelabrum4.png}

} %}}}

% Age of Progress

% Galileo 1610, Halley/Newton 1687{{{
\page{

\stagedir{Pegue a quarta vela, que representa a ci\^{e}ncia. Segure-a enquanto l\^{e} a
cita\c{c}\~{a}o, ent\~{a}o passe-a diretamente para a pr\'{o}xima pessoa. Isto se repete com
cada cita\c{c}\~{a}o nesta se\c{c}\~{a}o.}

\blockquote{Com a ajuda de um telesc\'{o}pio qualquer um pode observar isto, de uma
maneira que t\~{a}o distintamente apela aos sentidos, que todas as disputas que t\^{e}m
atormentado os fil\'{o}sofos por tantas eras s\~{a}o dirimidas de uma s\'{o} vez pela
evid\^{e}ncia indisput\'{a}vel de nossos olhos, e somos libertados de verbosas disputas
sobre este assunto; pois a Gal\'{a}xia nada mais \'{e} do que uma massa de inumer\'{a}veis
estrelas reunidas em grupos..}{Galileu Galilei, Mensagem das Estrelas (1610)}
% Footnote: Some words have been modernized. Sidereal->Starry, irrefragable->indisputable.

\divider

\blockquote{Assuntos que atormentavam a mente dos antigos s\'{a}bios,\newline
E que para nossos eruditos doutores sempre levavam\newline
A acaloradas e v\~{a}s contendas, agora s\~{a}o vistos\newline
\`{a} luz da raz\~{a}o, as nuvens da ignor\^{a}ncia\newline
Dispersadas enfim pela ci\^{e}ncia.  Aqueles sobre os quais\newline
A ilus\~{a}o lan\c{c}ava o sombrio manto da d\'{u}vida\newline
Soerguidos agora nas asas que o g\^{e}nio fornece\newline
Podem penetrar as mans\~{o}es dos deuses\newline
E escalar as alturas dos c\'{e}us. Erguei-vos,\newline
Mortais, e deixando de lado preocupa\c{c}\~{o}es terrenas\newline
Discerni o poder da mente celestial,\newline
Remota e distante da vida do gado!}{Edmund Halley, pref\'{a}cio aos 'Principia Mathematica' de Newton (1687)}

} %}}}


% Bayes 1763, Watt 1765 {{{
\page{
\blockquote{Por c\'{a}lculos semelhantes a estes podem ser determinadas de maneira
universal quais expectativas s\~{a}o justificadas por um experimento qualquer,
conforme o n\'{u}mero de vezes em que tenha falhado ou sido bem-sucedido; e o que
se deve pensar da probabilidade de que qualquer causa em particular na natureza,
com a qual tenhamos qualquer grau de familiaridade, produzir\'{a} ou n\~{a}o, num dado
evento, um efeito que tenha sido vinculado a ela.}{Rev. Thomas Bayes, 'Ensaio buscando resolver um problema na doutrina das probabilidades' (1763)}

\divider

\blockquote{Naquele tempo eu estava pensando no motor, e tinha chegado at\'{e} o
est\'{a}bulo quando a ideia veio \`{a} minha mente: como o vapor era um corpo el\'{a}stico,
ele buscaria ocupar um v\'{a}cuo, e se uma comunica\c{c}\~{a}o fosse feita entre o cilindro
e um recipiente esvaziado ele o ocuparia, e l\'{a} poderia se condensar sem resfriar
o cilindro. Ent\~{a}o eu vi que eu precisava me livrar do vapor condensado e da \'{a}gua
de inje\c{c}\~{a}o se eu usasse um jato, como no motor de Newcomen. Duas formas de fazer
isto me ocorreram... Eu n\~{a}o tinha sequer passado do chal\'{e} do golfe e a coisa
toda j\'{a} estava arranjada na minha mente.}{James Watt (1765)}

} %}}}

% Mendeleev 1864, Bell 1876, candle 5 {{{
\page{

\blockquote{Num sonho eu vi uma tabela, em que todos os elementos ca\'{\i}am nos seus
lugares como necess\'{a}rio. Ao despertar, eu imediatamente a anotei num papel, e
somente num ponto uma corre\c{c}\~{a}o pareceu necess\'{a}ria.}{Dmitri Mendel\^{e}iev (1864)}

\divider

\blockquote{Ent\~{a}o eu gritei no bocal a seguinte frase: 'Sr. Watson, venha aqui,
eu quero v\^{e}-lo'. Para minha alegria ele veio e declarou que havia ouvido e
entendido o que eu dissera. Eu lhe pedi para repetir as palavras. Ele respondeu:
``Voc\^{e} disse: Sr. Watson, venha aqui, eu quero v\^{e}-lo.''}{Alexander Graham Bell (1876)}

\divider

\blockquote{\'{E} sem exagero que eu digo que elaborei 3.000 hip\'{o}teses diferentes
em rela\c{c}\~{a}o \`{a} luz el\'{e}trica, todas elas razo\'{a}veis e aparentemente prov\'{a}veis de
serem verdadeiras. Mas apenas em dois casos os meus experimentos comprovaram a
verdade da minha hip\'{o}tese. Minha principal dificuldade foi em construir o
filamento de carbono... Todos os cantos do globo foram revirados pelos meus
agentes, e todo tipo de material ins\'{o}lito foi usado, at\'{e} que finalmente nos
decidimos pelo fiapo de bambu, que agora utilizamos.}{Thomas Edison (1890)}

} %}}}


% The urn of inventions {{{
\page{

\stagedir{Retorne a vela ao candelabro.}

Pare por um minuto para notar a escala de tempo destas descobertas. Cada uma
delas mudou significativamente a sociedade, e cada mudan\c{c}a foi, ao menos na
maior parte, para melhor.

\divider

\blockquote{Se n\'{o}s retirarmos continuamente amostras da urna de descobertas
tecnol\'{o}gicas poss\'{\i}veis, antes de implementarmos meios efetivos de coordena\c{c}\~{a}o,
vigil\^{a}ncia e/ou restri\c{c}\~{a}o globais a informa\c{c}\~{a}o potencialmente perigosa,
arriscamos acabar tirando uma bola preta: uma interven\c{c}\~{a}o f\'{a}cil de realizar e
que causa danos extremamente amplos, contra a qual a defesa efetiva \'{e}
impratic\'{a}vel.}{Nick Bostrom (2013)}

\divider

Entrando nas d\'{e}cadas de 30 e 40, muitas regras sobre as quais a sociedade humana
fora constru\'{\i}da haviam dado lugar \`{a} ci\^{e}ncia e \`{a} ind\'{u}stria. At\'{e} este ponto, o
progresso tecnol\'{o}gico avan\c{c}ava na velocidade da civiliza\c{c}\~{a}o, e seus efeitos
eram principalmente efeitos sobre sociedades. Cada tecnologia tinha um nome
associado a ela, mas esses nomes n\~{a}o importam muito.

}

%}}}

% Industrial era

% CUT: Industrialization {{{
%\page{
%
%With reason and learning, with lightning and steel, humans built on a scale
%never before seen. As we created tools and infrastructure, small gains in
%efficiency compounded, and compounded, and compounded again, leading to yet
%better tools and yet more infrastructure. We built tractors that feed cities,
%railroads that span continents, and economies that coordinate thousands of
%millions of humans.
%
%\blockquote{If a device would save in time just 10 per cent. or increase results 10 per
%cent., then its absence is always a 10 per cent. tax. If the time of a person
%is worth fifty cents an hour, a 10 per cent. saving is worth five cents an
%hour. ... Save ten steps a day for each of twelve thousand employees and you will
%have saved fifty miles of wasted motion and misspent energy. Those are the
%principles on which the production of my plant was built up.}{Henry Ford, ``My Life and Work'' (1922)}
%
%\divider
%
%} % }}}
% Political Science {{{
\page{
% Sneaky quote
\blockquoteUnattributed{Estas inven\c{c}\~{o}es materiais, come\c{c}ando com o uso de pedras
como armas, o que levou \`{a} domestica\c{c}\~{a}o de animais, a produ\c{c}\~{a}o do fogo por meios
artificiais, at\'{e} \`{a}s maravilhosas inven\c{c}\~{o}es de nossos dias, demonstram claramente
que em cada caso o iniciador foi um indiv\'{\i}duo. Quanto mais pr\'{o}ximos chegamos de
nossa pr\'{o}pria \'{e}poca e quanto mais importantes e revolucion\'{a}rias as inven\c{c}\~{o}es
ficam, mais claramente reconhecemos que isto \'{e} verdade. Todas as inven\c{c}\~{o}es
materiais que vemos ao nosso redor foram produzidas pelos poderes e capacidades
criativas de indiv\'{\i}duos.}

\divider

Cada um dos inventores mencionados at\'{e} aqui era uma pessoa basicamente boa,
interessada em encontrar a verdade, melhorar a sociedade ou, no pior dos casos,
criar um neg\'{o}cio para si mesmo. Newton dominou o c\'{a}lculo; Watt dominou o vapor;
Edison dominou a eletricidade. A hist\'{o}ria foi mudada por suas inven\c{c}\~{o}es, mas
n\~{a}o por seu car\'{a}ter.

%1939 no original
Mas em 1933, algu\'{e}m descobriu o segredo do \emph{poder} - o que hoje em dia
chamar\'{\i}amos de ci\^{e}ncia pol\'{\i}tica. Ele aprendeu como usar de maneira efetiva para
propaganda o cinema e o r\'{a}dio, quando estas inven\c{c}\~{o}es eram recentes. E, desta
vez, importa muito quem ele era. Foi ele quem escreveu a cita\c{c}\~{a}o anterior. E
hoje geralmente se considera que ele foi o pior homem que j\'{a} viveu.

}

% }}}


% World War 2 {{{
\page{

\blockquote{Quero dar \^{e}nfase ao fato de que o princ\'{\i}pio da democracia
parlamentar, no qual as decis\~{o}es s\~{a}o adotadas pelo voto da maioria, nem sempre
governou o mundo. Pelo contr\'{a}rio, encontramo-lo prevalente apenas durante
breves per\'{\i}odos da hist\'{o}ria, e estes sempre foram tempos de decl\'{\i}nio para
na\c{c}\~{o}es e Estados.}{Adolf Hitler, Mein Kampf (1926)}

%\blockquote{In those days the following happened almost always: I presented
%myself before an assembly of men who believed the opposite of what I wished to
%say and who wanted the opposite of what I believed in. Then I had to spend a
%couple of hours in persuading two or three thousand people to give up the
%opinions they had first held, in destroying the foundations of their views
%with one blow after another and finally in leading them over to take their
%stand on the grounds of our own convictions and our philosophy of life.
%
%I learned something that was important at that time, namely, to snatch from
%the hands of the enemy the weapons which he was using in his reply. I soon
%noticed that our adversaries, especially in the persons of those who led the
%discussion against us, were furnished with a definite repertoire of arguments
%out of which they took points against our claims which were being constantly
%repeated. The uniform character of this mode of procedure pointed to a
%systematic and unified training. And so we were able to recognize the
%incredible way in which the enemy's propagandists had been disciplined, and I
%am proud today that I discovered a means not only of making this propaganda
%ineffective but of beating the artificers of it at their own work. Two years
%later I was master of that art.}{Adolf Hitler, Mein Kampf (1926)}
%
%\blockquote{The art of propaganda lies in understanding the emotional ideas of
%the great masses and finding, through a psychologically correct form, the way
%to the attention and thence to the heart of the broad masses. ... There was no
%end to what could be learned from the enemy by a man who kept his eyes open,
%refused to let his perceptions be ossified, and for four and a half years
%privately turned the stormflood of enemy propaganda over in his brain.}{Adolf Hitler, Mein Kampf (1926)}

\divider

Come\c{c}ando em 1939 e indo at\'{e} 1945, a Segunda Guerra Mundial matou cerca de 60
milh\~{o}es de pessoas. Se o rumo da guerra tivesse sido diferente, \'{e} prov\'{a}vel que
o mundo inteiro tivesse ca\'{\i}do sob o dom\'{\i}nio de um \'{u}nico regime totalit\'{a}rio.
Isto n\~{a}o levaria exatamente \`{a} extin\c{c}\~{a}o humana - mas muito provavelmente
causaria a perda do potencial da humanidade.

Ent\~{a}o, as mentes mais brilhantes do planeta acreditaram que n\~{a}o tinham escolha.
Elas deviam se reunir em segredo e criar a bomba at\^{o}mica - uma arma para
arrasar cidades ou destruir o mundo inteiro.

} %}}}
% Manhattan Project: Take candle 5, and drip wax. {{{
\page{

\blockquote{Apesar da vis\~{a}o e da perspicaz sabedoria dos nossos chefes de
Estado durante a guerra, os f\'{\i}sicos se viram com a responsabilidade
peculiarmente \'{\i}ntima de sugerir, apoiar e, no fim das contas, em grande medida,
concretizar a constru\c{c}\~{a}o de armas at\^{o}micas. E n\~{a}o podemos nos esquecer de que
estas armas, tais como foram realmente usadas, dramatizaram de maneira impiedosa
a desumanidade e a malevol\^{e}ncia da guerra moderna. Em um certo sentido rasteiro
que nenhuma vulgaridade, nenhum humor, nenhum exagero podem realmente extinguir,
os f\'{\i}sicos conheceram o pecado; e este \'{e} um conhecimento que n\~{a}o podem perder.
}{J. Robert Oppenheimer (1947)}

%\stagedir{Take the fourth candle, which represents science. Hold it over the
%stack of papers until three drops of wax fall. Then use it to light the fifth
%candle, which represents industrialization.}
\stagedir{Usando a quarta vela, que representa a ci\^{e}ncia, acenda a quinta vela,
representando a industrializa\c{c}\~{a}o.}

\candelabrum{images/candelabrum5.png}

} % }}}


% Postwar Progress {{{
\page{

Ap\'{o}s a guerra, as coisas se acalmaram - mas o ritmo do progresso continuou.

Em 1951, o primeiro transistor.\newline
Em 1952, a primeira bomba de hidrog\^{e}nio.\newline
Em 1953, a descoberta da estrutura do DNA.\newline
Em 1954, a primeira c\'{e}lula solar, foguete miniaturizado, e submarino nuclear.\newline
Em 1955, a vacina contra a poliomielite.\newline
Em 1956, a primeira usina nuclear comercial.\newline
Em 1957, o Sputnik, o primeiro voo espacial a entrar em \'{o}rbita.\newline
Em 1958, o primeiro circuito integrado.\newline
Em 1959, o Lunik 2, o primeiro sat\'{e}lite a chegar \`{a} Lua.

\divider

Quando a nossa tecnologia decolou - literal e figurativamente - ningu\'{e}m sabia
o que \'{\i}amos encontrar. Predizer o futuro era algo que cabia principalmente a
autores de fic\c{c}\~{a}o cient\'{\i}fica, e suas previs\~{o}es n\~{a}o eram especialmente acuradas.
Mas alguns cientistas levaram as quest\~{o}es importantes a s\'{e}rio. Por exemplo:
haveria vida no espa\c{c}o? O programa SETI come\c{c}ou em 1961, num radio-observat\'{o}rio
na Virg\'{\i}nia Ocidental, EUA.

} % }}}
% Great Filter, Robin Hanson {{{
\page{

\blockquote{Eu escrevi todas as coisas que \'{e} preciso saber para predizer a
dificuldade de detectar vida extraterrestre. E observando-as ficou claro que,
ao multiplic\'{a}-las todas juntas, se obtinha um n\'{u}mero N, que corresponde ao
n\'{u}mero de civiliza\c{c}\~{o}es detect\'{a}veis em nossa gal\'{a}xia.}{Frank Drake}

\divider

\blockquote{A humanidade parece ter um futuro brilhante, isto \'{e}, uma chance n\~{a}o
trivial de se expandir para preencher o universo com vida duradoura. Mas o fato
de o espa\c{c}o na nossa vizinhan\c{c}a parecer morto at\'{e} agora nos diz que qualquer
peda\c{c}o de mat\'{e}ria inerte tem diante de si uma chance astronomicamente pequena
de criar um futuro como esse. Portanto, existe um grande filtro entre a morte e
a expans\~{a}o duradoura da vida, e a humanidade se depara com a quest\~{a}o amea\c{c}adora:
em que ponto desse filtro n\'{o}s estamos?}{Robin Hanson (1998)}

%\divider
%
%The absence of observed alien life, taken together with our current best
%estimates of how unlikely it was for life to arise and to advance to the point
%where we are now, means that there must be some filter further in our future,
%which prevents most life from spreading through the stars. Therefore, either
%our model of the past or of the universe is very substantially wrong, or there
%is an obstacle in our future which most intelligent civilizations fail to pass.
%We have several ideas about what that obstacle might be.

} % }}}


% CUT: Dartmouth Conference 1956 {{{
%\page{
%
%One thing we did \textit{not} invent in the fifties, was artificial
%intelligence. But not entirely for lack of trying. Two years before the first
%integrated circuit and 15 years before the first microprocessor, John McCarthy
%organized the Dartmouth conference to study the problem of artificial
%intelligence.
%
%\divider
%
%\blockquote{We propose that a 2 month, 10 man study of artificial intelligence
%be carried out during the summer of 1956 at Dartmouth College in Hanover, New
%Hampshire.  The study is to proceed on the basis of the conjecture that every
%aspect of learning or any other feature of intelligence can in principle be so
%precisely described that a machine can be made to simulate it. An attempt will
%be made to find how to make machines use language, form abstractions and
%concepts, solve kinds of problems now reserved for humans, and improve
%themselves. We think that a significant advance can be made in one or more of
%these problems if a carefully selected group of scientists work on it together
%for a summer.}{McCarthy et al, 1956}
%
%} % }}}
% CUT: SETI 1961, Drake equation {{{
%\page{
%
%In 1961, the Search for Extraterrestrial Intelligence (SETI) held its first
%conference at the National Radio Astronomy Observatory in West Virginia, to
%discuss what is now called the Drake Equation:
%
%In forty-odd years of searching, we have found no extraterrestrial life. Our
%best estimates for the Drake equation, however, suggest that there should be.
%This is the Fermi Paradox: where is everybody?
%
%} % }}}


% Cuban Missile Crisis 1962: Drip wax {{{
\page{
Em 1962, a Guerra Fria entre a Uni\~{a}o Sovi\'{e}tica e os Estados Unidos levou a uma
crise. Destr\'{o}ieres americanos com ordens de impor uma quarentena naval na costa
cubana n\~{a}o sabiam que os submarinos que os sovi\'{e}ticos haviam enviado para
proteger seus navios portavam armas nucleares. Ent\~{a}o os americanos come\c{c}aram a
disparar cargas de profundidade, para for\c{c}ar os submarinos a vir \`{a} tona, uma
a\c{c}\~{a}o que os sovi\'{e}ticos a bordo interpretaram como o come\c{c}o da Terceira Guerra
Mundial.

\divider

\blockquote{Vamos explodi-los agora! N\'{o}s vamos morrer, mas vamos afundar todos
eles. N\~{a}o vamos desonrar nossa marinha!}{Capit\~{a}o Valentin Grigori\'{e}vitch Savitsky (1962)}

\stagedir{Pegue a quinta vela, que representa a ind\'{u}stria. Segure-a sobre a
pilha de pap\'{e}is no centro at\'{e} que tr\^{e}s gotas de cera caiam.}

} % }}}
% Arkhipov {{{
\page {
O lan\c{c}amento do torpedo nuclear do submarino exigia o consenso dos tr\^{e}s oficiais
superiores a bordo: o capit\~{a}o Valentin Grigori\'{e}vitch Savitsky, o oficial
pol\'{\i}tico Ivan Semion\'{o}vitch, e o imediato Vass\'{\i}li Arkhipov.

Arkhipov foi o \'{u}nico a se negar a lan\c{c}ar a arma at\^{o}mica, insistindo em que o
submarino viesse \`{a} tona para receber ordens de Moscou. Se ele houvesse tomado
uma decis\~{a}o diferente, o resultado provavelmente teria sido uma guerra nuclear
total.

\begin{center}
	\includegraphics[width=3.0in]{images/VasiliArkhipov.jpg}
\end{center}

} % }}}


% 1963: IJ Good, Superintelligence: Place an unlit candle in the last spot {{{
\page{

Enquanto isso, a tecnologia seguia sua marcha. E, pela primeira vez, surgiu
a ideia de que o progresso tecnol\'{o}gico poderia n\~{a}o seguir indefinidamente, mas
sim culminar numa realiza\c{c}\~{a}o definitiva.

\divider

\blockquote{Seja uma m\'{a}quina ultrainteligente definida como uma m\'{a}quina que
ultrapassa, em muito, todas as atividades intelectuais de qualquer ser humano,
por mais inteligente que este seja. Como a constru\c{c}\~{a}o de m\'{a}quinas \'{e} uma destas
atividades intelectuais, uma m\'{a}quina ultrainteligente pode criar m\'{a}quinas ainda
melhores; inquestionavelmente haveria ent\~{a}o uma ``explos\~{a}o de intelig\^{e}ncia'', e
a intelig\^{e}ncia humana seria deixada para tr\'{a}s. Assim, a primeira m\'{a}quina
ultrainteligente \'{e} a \'{u}ltima inven\c{c}\~{a}o que o homem precisa fazer, desde que a
m\'{a}quina seja d\'{o}cil o bastante para nos dizer como mant\^{e}-la sob controle.
}{I.J. Good, Especula\c{c}\~{o}es a Respeito da Primeira M\'{a}quina Ultrainteligente (1963)}

\stagedir{Coloque uma vela apagada na \'{u}ltima posi\c{c}\~{a}o, para representar a
tecnologia futura.}

\candelabrum{images/candelabrum5b.png}

} % }}}
% Moore's Law 1965: Light candle 6, Computers {{{
\page{

Dois anos depois, Gordon Moore fez a famosa observa\c{c}\~{a}o:

\blockquote{A complexidade de componentes de custo m\'{\i}nimo tem aumentado ao ritmo
de um fator de cerca de dois a cada ano. Certamente, no curto prazo, pode-se
esperar que este ritmo continuar\'{a}, ou mesmo se acelerar\'{a}. Num prazo mais
dilatado, a taxa de crescimento \'{e} um pouco mais incerta, embora n\~{a}o haja raz\~{a}o
para crer que ela n\~{a}o permanecer\'{a} praticamente constante por pelo menos 10 anos.
Isto quer dizer que em 1975 o n\'{u}mero de componentes por circuito integrado de
custo m\'{\i}nimo ser\'{a} de 65.000.}{Gordon Moore (1965)}

\vspace*{1.2in}

\stagedir{Usando a quinta vela, que representa a industrializa\c{c}\~{a}o, acenda a
sexta vela, para representar a inven\c{c}\~{a}o do computador.}

\candelabrum{images/candelabrum6.png}

%\divider
%
%Moore's Law has been rephrased many times, to talk about slightly different
%things - clock rate, computation speed, cost per computation speed, cost per
%computation. Some of these phrasings must eventually face fundamental physical
%limits - but some of them need not. For now, the speed of computation is
%increasing at the slightly slower rate of roughly a factor of two every other
%year. It shows no sign of stopping.

} % }}}


% Moore's Law of Mad Science; Moon landing {{{
\page {

\blockquote{Lei de Moore dos Cientistas Malucos: a cada 18 meses, o QI
necess\'{a}rio para destruir o mundo diminui 1 ponto.}{Fonte desconhecida (2005)}

\divider

Enquanto isso, os foguetes que haviam sido criados para a guerra foram
encaminhados tamb\'{e}m para a explora\c{c}\~{a}o:

\blockquote{Aqui, homens do planeta Terra primeiro pisaram na Lua\newline
Julho de 1969 A.D. - Viemos em paz por toda a humanidade}{Placa da Apollo 11 (1969)}

} % }}}
% CUT: AI Winter {{{
%\page{
%
%Compared to landing on the moon, computers were just not that impressive. In
%1973 the British Science Research Council commissioned James Lighthill to
%evaluate the state of artificial intelligence:
%
%\blockquote{Most workers in AI research and in related fields confess to a
%pronounced feeling of disappointment in what has been achieved in the past
%twenty-five years. Workers entered the field around 1950, and even around 1960,
%with high hopes that are very far from having been realized in 1972. In no part
%of the field have the discoveries made so far produced the major impact that
%was then promised.}{James Lighthill (1973)}
%
%\divider
%
%So began the AI winter: a period of pessimism, reduced funding, and minimal
%progress. We now know that AI is hard; how hard exactly, we don't know. Since
%then, Moore's Law has continued - and so, every year, the difficulty drops a
%little bit.
%
%} % }}}
% Population growth {{{
\page{

Para que n\~{a}o nos esque\c{c}amos o quanto \'{e} dif\'{\i}cil predizer o futuro, eis um
desastre previsto que n\~{a}o se concretizou.

\divider

\blockquote{A batalha para alimentar toda a humanidade est\'{a} perdida. Nos anos
1970 e 80, centenas de milh\~{o}es de pessoas v\~{a}o morrer de fome a despeito de
quaisquer programas de emerg\^{e}ncia que se iniciem agora. Neste ponto t\~{a}o tardio,
nada pode impedir um aumento substancial na taxa mundial de mortalidade, embora
muitas vidas possam ser salvas por programas dram\'{a}ticos destinados a "esticar"
a capacidade de sustento do planeta por meio do aumento da produ\c{c}\~{a}o de alimentos
e uma distribui\c{c}\~{a}o mais equitativa da comida dispon\'{\i}vel. Mas estes programas s\'{o}
adiariam a cat\'{a}strofe a menos que tamb\'{e}m se empreenda um determinado e eficaz
controle populacional. O controle populacional \'{e} a regula\c{c}\~{a}o consciente do
n\'{u}mero de seres humanos para atender n\~{a}o somente \`{a}s necessidades das fam\'{\i}lias,
mas tamb\'{e}m da sociedade como um todo.
\newline\newline

Nada poderia ser mais enganoso para nossos filhos que a nossa atual sociedade
abastada. Eles herdar\~{a}o um mundo totalmente diferente, um mundo em que os
padr\~{o}es, a pol\'{\i}tica e a economia da \'{u}ltima d\'{e}cada estar\~{a}o mortos. Sendo a na\c{c}\~{a}o
mais influente do mundo atual, e seu maior consumidor, os Estados Unidos n\~{a}o
podem ficar isolados. Hoje, estamos envolvidos nos eventos que est\~{a}o levando \`{a}
fome e \`{a} ecocat\'{a}strofe; amanh\~{a} podemos ser destru\'{\i}dos por eles.}{Paul Ehrlich (1968)}

} %}}}


% CUT: Flynn Effect {{{
%\page{
%At this point, it's worth noting another effect, something like Moore's Law,
%which also continues year after year: the Flynn effect. Ever since mental tests
%were introduced, the mean IQ of Americans has been rising at a rate of about .3
%points per year. I went to James Flynn's paper on this, published in 1984,
%expecting to find a discussion of advancing intelligence, pick out an
%illustrative quote and move on. That is not what I found.
%
%While technology changes rapidly, it is often said that human nature doesn't
%change. Rising IQs are a change to human nature, but a comprehensible one; it
%is mostly a matter of the average people becoming more like the best people.
%But human nature changes in more ways than one, and while Flynn was interested
%in rising IQ test results, his goal was to reconcile them with falling scores
%on a different test, the Scholastic Aptitude Test or SAT.
%
%\divider
%
%\blockquote{Our analysis dictates the conclusion that American 18-year-olds have
%deteriorated .48 to .96 SDU in terms of a total package consisting of
%motivation, study habits, self-discipline, and acquired verbal and writing
%skills. Which is to say that only the upper, 17\% to 32\% of today's 18-year-olds
%can match the upper half of young people as recently as 1963! The calculations
%above should not be taken literally of course: They are merely meant to show
%that if both IQ gains and SAT losses are taken to be real, rather than
%artifacts of sampling error, then the deterioration of non-IQ personal traits
%among young Americans must have been very great.}{James R. Flynn (1984)}
%
%\divider
%
%I don't know how to compare the personality traits of populations over time,
%but I do know this: the world is suddenly filled with stressors and drugs and
%superstimuli that enable humans to go insane in a variety of novel and exciting
%ways, and we keep inventing new ones. Sometimes, humans go insane in ways that
%are dangerous to those around them, and sometimes they go insane from positions
%of power.
%
%} % }}}

% Petrov incident - drip wax {{{
\pageNoBottomDiv{
Deixando de lado as tend\^{e}ncias de longo prazo e voltando aos eventos concretos,
chegamos ao momento hist\'{o}rico que d\'{a} nome ao dia de hoje: o incidente Petrov.
No dia 26 de setembro de 1983, Stanislav Petrov era o oficial de plant\~{a}o no
Oko, o sistema sovi\'{e}tico de alerta r\'{a}pido contra ataques nucleares.

\divider

\blockquote{Um alarme disparou no posto de comando e controle, com luzes
vermelhas piscando no terminal. Foi um choque terr\'{\i}vel. Todos pularam de suas
cadeiras, olhando para mim. O que eu podia fazer? Havia um procedimento
operacional que eu mesmo havia escrito. N\'{o}s fizemos o que t\'{\i}nhamos que fazer.
Conferimos a opera\c{c}\~{a}o de todos os sistemas - em 30 n\'{\i}veis, um depois do outro.
Os relat\'{o}rios continuavam a chegar: tudo est\'{a} correto; o fator de probabilidade
\'{e} 2... o mais alto.}{Stanislav Petrov}

%\stagedir{Take the fifth candle, which represents industry. Hold it over
%the stack of papers until three drops of wax fall.}

\divider

\begin{center}
	\includegraphics[width=3.0in]{images/StanislavPetrov.jpg}
\end{center}

} % }}}
% Petrov incident {{{
\page{
\blockquote{Eu me perguntei se eu assumiria a responsabilidade por iniciar a
Terceira Guerra Mundial - e eu disse n\~{a}o, n\~{a}o assumo... Eu sempre pensava
nisso. Sempre que come\c{c}ava meu turno, eu refrescava isso na minha mem\'{o}ria.}{Stanislav Petrov}

\divider

Se ele tivesse seguido o procedimento e relatado para seus superiores que os
americanos tinham lan\c{c}ado m\'{\i}sseis, isto podia ter iniciado uma guerra nuclear.
Ent\~{a}o, ao inv\'{e}s de encaminhar o alarme na cadeia de comando, ele disse que se
tratava de um alarme falso - mesmo sem saber ao certo o que estava acontecendo.

\divider

%\blockquoteUnattributed{You can't possibly analyze things properly within a
%couple of minutes ... All you can rely on is your intuition. I had two
%arguments to fall back on. First, missile attacks do not start from just one
%base. Second, the computer is, by definition, brainless. There are lots of
%things it can mistake for a missile launch.}

Na \'{e}poca, ele n\~{a}o recebeu pr\^{e}mio algum. O incidente foi um vexame para seus
superiores e para os cientistas respons\'{a}veis pelo sistema, ent\~{a}o, se ele fosse
recompensado, eles teriam que ser punidos. (Ele recebeu o Pr\^{e}mio Internacional
da Paz trinta anos depois, em 2013.

As coisas acabaram se acalmando. A Uni\~{a}o Sovi\'{e}tica foi dissolvida. Salvaguardas
foram instaladas na maioria das bombas, para impedir o risco de detona\c{c}\~{a}o
acidental (ou deliberada mas n\~{a}o autorizada).

} %}}}

%%%%%%%%%%%%%%%%%%%%%%%%%%%%%%%%%%%%%%%%%%%%%%%%%%%%%%%%%%%%%%%%%%%%%%%%%%%%%%%

% Ozone layer {{{
\page{
Em 1985, Joe Farman, Brian Gardiner e Jonathan Shanklin tiveram fizeram uma
descoberta inquietante. A camada de oz\^{o}nio, a parte da atmosfera que bloqueia
a maior parte da radia\c{c}\~{a}o ultravioleta, estava desaparecendo devido \`{a} polui\c{c}\~{a}o
por clorofluorcarbonos. Em apenas dois anos, um tratado foi assinado proibindo
o uso de CFCs, e em mais dois anos, em 1989, ele estava em vigor. Atualmente,
todos os pa\'{\i}ses-membros das Na\c{c}\~{o}es Unidas ratificaram o Protocolo de Montreal.

\divider

\blockquote{O buraco na camada de oz\^{o}nio \'{e} uma esp\'{e}cie de mensagem no c\'{e}u. No
come\c{c}o ele parecia significar a nossa complac\^{e}ncia constante diante de um
caldeir\~{a}o de bruxa de perigos mortais. Mas talvez ele queira dizer, na verdade,
que rec\'{e}m descobrimos um talento para trabalhar em conjunto para proteger o meio
ambiente global.}{Carl Sagan (1998)}

} % }}}

% AI risk {{{
\page{

Mas nem toda amea\c{c}a \`{a} humanidade \'{e} como as armas nucleares ou o buraco na camada
de oz\^{o}nio, relativamente simples de entender e resolver.

\divider

\blockquote{Uma IA n\~{a}o Amig\'{a}vel com nanotecnologia molecular (ou outra forma
de infraestrutura r\'{a}pida) n\~{a}o precisa se preocupar em marchar ex\'{e}rcitos ou
chantagear ou praticar coer\c{c}\~{a}o econ\^{o}mica sutil. A IA n\~{a}o Amig\'{a}vel tem a
capacidade de reorganizar toda a mat\'{e}ria no sistema solar conforme seu alvo de
otimiza\c{c}\~{a}o. Isto \'{e} fatal para n\'{o}s, se a IA n\~{a}o escolher especificamente de
acordo com o crit\'{e}rio de como esta transforma\c{c}\~{a}o afeta padr\~{o}es existentes como
a biologia e as pessoas. A IA n\~{a}o te odeia, e nem te ama, mas voc\^{e} \'{e} feito de
\'{a}tomos que ela pode usar para algum outro fim. Ela opera numa escala de tempo
diferente da sua; at\'{e} seus neur\^{o}nios terminarem de pensar as palavras ``eu
devia fazer alguma coisa'', voc\^{e} j\'{a} perdeu.}{Eliezer Yudkowsky, Intelig\^{e}ncia
Artificial como um Fator Positivo e Negativo para o Risco Global (2006)}
% perdeu, preib\'{o}i

%\divider
%
%At the Global Catastrophic Risk Conference in Oxford (17-20 July, 2008) an
%informal survey was circulated among participants, asking them to make their
%best guess at the chance that there will be disasters of different types before
%2100. The median estimate was a 19\% chance of human extinction.

} % }}}
% Differential Technological Advancement, AI Progress {{{
\page{

\blockquote{O que n\'{o}s temos o poder de afetar (e a medida disso depende de como
definirmos ``n\'{o}s'') \'{e} o ritmo de desenvolvimento de diversas tecnologias e,
potencialmente, a sequ\^{e}ncia em que tecnologias poss\'{\i}veis s\~{a}o desenvolvidas e
implementadas. Nosso foco deveria ser no que eu quero chamar de desenvolvimento
tecnol\'{o}gico diferencial: tentar retardar a implementa\c{c}\~{a}o de tecnologias
perigosas e acelerar a implementa\c{c}\~{a}o de tecnologias ben\'{e}ficas, especialmente
aquelas que aliviam os riscos causados por outras tecnologias.}{Nick Bostrom (2002)}

\stagedir{Coloque uma vela apagada no pen\'{u}ltimo lugar, para representar futuros
alternativos poss\'{\i}veis.}

\candelabrum{images/candelabrum6b.png}

} %}}}

% AI risk continued {{{
% TODO: This page is weak
\page{

\blockquote{Por um lado, eu acho que teria tido boas chances numa revanche em
1998 se estivesse mais bem-preparado; por outro, naquela \'{e}poca j\'{a} estava claro
que a superioridade dos computadores sobre os humanos no xadrez sempre fora uma
quest\~{a}o de tempo.}{Garry Kasparov, campe\~{a}o mundial de xadrez, ap\'{o}s perder para
o computador Deep Blue, da IBM}

\divider

Vinte e quatro anos depois que o Relat\'{o}rio Lighthill declarou a IA um fracasso,
em 1997 o programa de computador Deep Blue derrotou o campe\~{a}o mundial de xadrez,
Garry Kasparov. Como se descobriu, o xadrez n\~{a}o era t\~{a}o dif\'{\i}cil como pens\'{a}vamos.
A intelig\^{e}ncia totalmente geral, no entanto, continuava inating\'{\i}vel.

\divider

\blockquote{Eu, pelo menos, dou as boas-vindas aos nossos novos dominadores,
os computadores.}{Ken Jennings, campe\~{a}o do programa de perguntas e respostas
Jeopardy, ap\'{o}s perder para o computador Watson, da IBM}

Como se descobriu em 2011, Jeopardy n\~{a}o era t\~{a}o dif\'{\i}cil quanto pens\'{a}vamos. Mas
a intelig\^{e}ncia totalmente geral permanece inating\'{\i}vel. 

N\'{o}s achamos que ela \'{e} dif\'{\i}cil.

} %}}}
% AI risk 3 {{{
% TODO: Rewrite this page
\page{

Ser\'{a} que o progresso na computa\c{c}\~{a}o realmente nos amea\c{c}a? At\'{e} aqui, sempre que a
ci\^{e}ncia e a tecnologia avan\c{c}aram, o florescimento da humanidade avan\c{c}ou \textit{pari
passu}. N\'{o}s constru\'{\i}mos horrores, \'{e} certo: metralhadoras e g\'{a}s mostarda e mesmo
bombas at\^{o}micas. Mas seu impacto agregado na vida humana \'{e} min\'{u}sculo comparado ao
da avia\c{c}\~{a}o, das telecomunica\c{c}\~{o}es, dos antibi\'{o}ticos e de dez mil outros milagres.

Talvez a intelig\^{e}ncia artificial tamb\'{e}m ser\'{a} tornada segura, mas o exemplo das
armas nucleares mostra que isso n\~{a}o \'{e} uma certeza. Se n\~{a}o fosse pelas a\c{c}\~{o}es de
Arkhipov e Petrov, poder\'{\i}amos ter eliminado n\~{a}o somente n\'{o}s mesmos, mas os
filhos de nossos filhos e a possibilidade de jamais alcan\c{c}ar o que est\'{a} al\'{e}m da
Terra.

O que nos leva \`{a} nossa pr\'{o}xima crise, em 2012, e esta n\~{a}o \'{e} t\~{a}o clara.

} % }}}

% A/H5N1 NSABB report {{{
\page{
\blockquote{Recentemente, diversas equipes de pesquisa cient\'{\i}fica conseguiram
algum sucesso em modificar v\'{\i}rus de gripe A/H5N1 para serem transmitidos de
maneira eficiente entre mam\'{\i}feros, e em um caso foi mantida a alta
patogenicidade... O NSABB foi un\^{a}nime em declarar que a comunica\c{c}\~{a}o dos
resultados dos dois manuscritos submetidos deveria ser fortemente limitada em
termos dos detalhes e resultados dos experimentos. As ci\^{e}ncias da vida chegaram
a uma encruzilhada. A dire\c{c}\~{a}o que vamos escolher e o processo pelo qual vamos
adot\'{a}-la devem ser da comunidade e n\~{a}o de pequenos segmentos do governo, da
comunidade cient\'{\i}fica ou da sociedade. F\'{\i}sicos se defrontaram com uma situa\c{c}\~{a}o
parecida na d\'{e}cada de 1940 com a pesquisa de armas nucleares, e \'{e} inevit\'{a}vel
que outras disciplinas cient\'{\i}ficas tamb\'{e}m passar\~{a}o por isso.}{NSABB - Conselho
Consultivo de Seguran\c{c}a Nacional em Biosseguran\c{c}a (2012)}

\divider

Ap\'{o}s v\'{a}rios meses, a decis\~{a}o foi alterada, e uma vers\~{a}o revisada do artigo da
gripe avi\'{a}ria foi aprovada para publica\c{c}\~{a}o, por 12 votos a 6.

\divider

E agora chegamos ao presente. At\'{e} aqui, a humanidade nem se destruiu nem se
colocou numa posi\c{c}\~{a}o segura. Mas este \'{e} apenas o meio da est\'{o}ria. Estamos nos
aproximando do cl\'{\i}max da hist\'{o}ria humana, onde iremos ou nos aniquilar ou nos
espalhar pelas estrelas.

} %}}}

% CUT: Differential intellectual progress 2 {{{
%\page{
%
%\blockquote{Differential intellectual progress consists in prioritizing risk-reducing
%intellectual progress over risk-increasing intellectual progress. As applied to
%AI risks in particular, a plan of differential intellectual progress would
%recommend that our progress on the scientific, philosophical, and technological
%problems of AI safety outpace our progress on the problems of AI capability
%such that we develop safe superhuman AIs before we develop (arbitrary)
%superhuman AIs. Our first superhuman AI must be a safe superhuman AI, for we
%may not get a second chance. With AI as with other technologies, we may become
%victims of ``the tendency of technological advance to outpace the social control
%of technology''.}{Luke Muehlhauser, Anna Salamon (2012)}
%
%\stagedir{Place an unlit candle in the eighth spot, to represent safe artificial intelligence.}
%
%}
% }}}

%%%%%%%%%%%%%%%%%%%%%%%%%%%%%%%%%%%%%%%%%%%%%%%%%%%%%%%%%%%%%%%%%%%%%%%%%%%%%%%

% Ending

% By the power of these candles... {{{
\page{

\stagedir{Pegue a primeira vela. Leia o que se segue, e a reponha.}
\blockspacing{Pelo poder do fogo eu sou dotado de vida, libertado das demandas
da mera subsist\^{e}ncia, e capaz de me importar com o futuro.}

\divider

\stagedir{Pegue a segunda vela. Leia o que se segue, e a reponha.}
\blockspacing{Pelo poder da linguagem, eu sou capaz de dividir o que sei do
mundo e do futuro, e ouvir o que outros aprenderam antes de mim.}

\divider

\stagedir{Pegue a terceira vela. Leia o que se segue, e a reponha.}
\blockspacing{Pelo poder da escrita, minhas palavras ecoam pelo tempo e espa\c{c}o.
Eu participo de um di\'{a}logo de bilh\~{o}es, e juntos escolhemos o futuro que queremos.}

} %}}}
% By the power of these candles... {{{
\page{

\stagedir{Pegue a quarta vela. Leia o que se segue, e a reponha.}
\blockspacing{Pelo poder da ci\^{e}ncia, eu conhe\c{c}o a verdadeira natureza do mundo
em que vivo. Eu vivo num mundo que obedece a leis naturais, que governam os
resultados de minhas a\c{c}\~{o}es, e por isso eu posso saber as consequ\^{e}ncias do que
fa\c{c}o.}

\divider

\stagedir{Pegue a quinta vela. Leia o que se segue, e a reponha.}
\blockspacing{Pelo poder da ind\'{u}stria, o mundo em que eu vivo \'{e} transformado.
Eu posso transform\'{a}-lo mais ainda, se desejar.}

\divider

\stagedir{Pegue a sexta vela e leia o que se segue.}
\blockspacing{Pelo poder da computa\c{c}\~{a}o, o poder da minha mente \'{e} amplificado.
Eu posso ver a totalidade do conhecimento da humanidade, um grandioso padr\~{a}o
fractal de sum\'{a}rios e detalhes e sum\'{a}rios de sum\'{a}rios e detalhes de detalhes,
e posso procurar em tudo isso com apenas uma palavra.}
\stagedir{Segure a vela junto da s\'{e}tima e oitava velas, que representam os
futuros bons e ruins. Ent\~{a}o afaste-a e retorne-a ao seu lugar.}

} %}}}

% We choose... {{{
\page{

\stagedir{Leiam o seguinte, juntos.}

Cada um de n\'{o}s, como indiv\'{\i}duos, deve escolher:\newline
Vamos ficar de lado enquanto a hist\'{o}ria continua?\newline
Vamos ser her\'{o}is, nos movendo para o centro da a\c{c}\~{a}o?\newline
Vamos encontrar os her\'{o}is e ser seus aliados, de perto ou \`{a} dist\^{a}ncia?\newline
\'{e} f\'{a}cil nos imaginarmos impotentes, e usar isso como desculpa.\newline
\newline
N\~{a}o nos esque\c{c}amos do poder que temos, e do que realmente somos.

} %}}}

\pagenumbering{gobble}

\end{document}
